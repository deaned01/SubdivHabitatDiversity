% Options for packages loaded elsewhere
\PassOptionsToPackage{unicode}{hyperref}
\PassOptionsToPackage{hyphens}{url}
%
\documentclass[
]{article}
\title{Worked examples}
\author{}
\date{\vspace{-2.5em}}

\usepackage{amsmath,amssymb}
\usepackage{lmodern}
\usepackage{iftex}
\ifPDFTeX
  \usepackage[T1]{fontenc}
  \usepackage[utf8]{inputenc}
  \usepackage{textcomp} % provide euro and other symbols
\else % if luatex or xetex
  \usepackage{unicode-math}
  \defaultfontfeatures{Scale=MatchLowercase}
  \defaultfontfeatures[\rmfamily]{Ligatures=TeX,Scale=1}
\fi
% Use upquote if available, for straight quotes in verbatim environments
\IfFileExists{upquote.sty}{\usepackage{upquote}}{}
\IfFileExists{microtype.sty}{% use microtype if available
  \usepackage[]{microtype}
  \UseMicrotypeSet[protrusion]{basicmath} % disable protrusion for tt fonts
}{}
\makeatletter
\@ifundefined{KOMAClassName}{% if non-KOMA class
  \IfFileExists{parskip.sty}{%
    \usepackage{parskip}
  }{% else
    \setlength{\parindent}{0pt}
    \setlength{\parskip}{6pt plus 2pt minus 1pt}}
}{% if KOMA class
  \KOMAoptions{parskip=half}}
\makeatother
\usepackage{xcolor}
\IfFileExists{xurl.sty}{\usepackage{xurl}}{} % add URL line breaks if available
\IfFileExists{bookmark.sty}{\usepackage{bookmark}}{\usepackage{hyperref}}
\hypersetup{
  pdftitle={Worked examples},
  hidelinks,
  pdfcreator={LaTeX via pandoc}}
\urlstyle{same} % disable monospaced font for URLs
\usepackage[margin=1in]{geometry}
\usepackage{color}
\usepackage{fancyvrb}
\newcommand{\VerbBar}{|}
\newcommand{\VERB}{\Verb[commandchars=\\\{\}]}
\DefineVerbatimEnvironment{Highlighting}{Verbatim}{commandchars=\\\{\}}
% Add ',fontsize=\small' for more characters per line
\usepackage{framed}
\definecolor{shadecolor}{RGB}{248,248,248}
\newenvironment{Shaded}{\begin{snugshade}}{\end{snugshade}}
\newcommand{\AlertTok}[1]{\textcolor[rgb]{0.94,0.16,0.16}{#1}}
\newcommand{\AnnotationTok}[1]{\textcolor[rgb]{0.56,0.35,0.01}{\textbf{\textit{#1}}}}
\newcommand{\AttributeTok}[1]{\textcolor[rgb]{0.77,0.63,0.00}{#1}}
\newcommand{\BaseNTok}[1]{\textcolor[rgb]{0.00,0.00,0.81}{#1}}
\newcommand{\BuiltInTok}[1]{#1}
\newcommand{\CharTok}[1]{\textcolor[rgb]{0.31,0.60,0.02}{#1}}
\newcommand{\CommentTok}[1]{\textcolor[rgb]{0.56,0.35,0.01}{\textit{#1}}}
\newcommand{\CommentVarTok}[1]{\textcolor[rgb]{0.56,0.35,0.01}{\textbf{\textit{#1}}}}
\newcommand{\ConstantTok}[1]{\textcolor[rgb]{0.00,0.00,0.00}{#1}}
\newcommand{\ControlFlowTok}[1]{\textcolor[rgb]{0.13,0.29,0.53}{\textbf{#1}}}
\newcommand{\DataTypeTok}[1]{\textcolor[rgb]{0.13,0.29,0.53}{#1}}
\newcommand{\DecValTok}[1]{\textcolor[rgb]{0.00,0.00,0.81}{#1}}
\newcommand{\DocumentationTok}[1]{\textcolor[rgb]{0.56,0.35,0.01}{\textbf{\textit{#1}}}}
\newcommand{\ErrorTok}[1]{\textcolor[rgb]{0.64,0.00,0.00}{\textbf{#1}}}
\newcommand{\ExtensionTok}[1]{#1}
\newcommand{\FloatTok}[1]{\textcolor[rgb]{0.00,0.00,0.81}{#1}}
\newcommand{\FunctionTok}[1]{\textcolor[rgb]{0.00,0.00,0.00}{#1}}
\newcommand{\ImportTok}[1]{#1}
\newcommand{\InformationTok}[1]{\textcolor[rgb]{0.56,0.35,0.01}{\textbf{\textit{#1}}}}
\newcommand{\KeywordTok}[1]{\textcolor[rgb]{0.13,0.29,0.53}{\textbf{#1}}}
\newcommand{\NormalTok}[1]{#1}
\newcommand{\OperatorTok}[1]{\textcolor[rgb]{0.81,0.36,0.00}{\textbf{#1}}}
\newcommand{\OtherTok}[1]{\textcolor[rgb]{0.56,0.35,0.01}{#1}}
\newcommand{\PreprocessorTok}[1]{\textcolor[rgb]{0.56,0.35,0.01}{\textit{#1}}}
\newcommand{\RegionMarkerTok}[1]{#1}
\newcommand{\SpecialCharTok}[1]{\textcolor[rgb]{0.00,0.00,0.00}{#1}}
\newcommand{\SpecialStringTok}[1]{\textcolor[rgb]{0.31,0.60,0.02}{#1}}
\newcommand{\StringTok}[1]{\textcolor[rgb]{0.31,0.60,0.02}{#1}}
\newcommand{\VariableTok}[1]{\textcolor[rgb]{0.00,0.00,0.00}{#1}}
\newcommand{\VerbatimStringTok}[1]{\textcolor[rgb]{0.31,0.60,0.02}{#1}}
\newcommand{\WarningTok}[1]{\textcolor[rgb]{0.56,0.35,0.01}{\textbf{\textit{#1}}}}
\usepackage{graphicx}
\makeatletter
\def\maxwidth{\ifdim\Gin@nat@width>\linewidth\linewidth\else\Gin@nat@width\fi}
\def\maxheight{\ifdim\Gin@nat@height>\textheight\textheight\else\Gin@nat@height\fi}
\makeatother
% Scale images if necessary, so that they will not overflow the page
% margins by default, and it is still possible to overwrite the defaults
% using explicit options in \includegraphics[width, height, ...]{}
\setkeys{Gin}{width=\maxwidth,height=\maxheight,keepaspectratio}
% Set default figure placement to htbp
\makeatletter
\def\fps@figure{htbp}
\makeatother
\setlength{\emergencystretch}{3em} % prevent overfull lines
\providecommand{\tightlist}{%
  \setlength{\itemsep}{0pt}\setlength{\parskip}{0pt}}
\setcounter{secnumdepth}{-\maxdimen} % remove section numbering
\ifLuaTeX
  \usepackage{selnolig}  % disable illegal ligatures
\fi

\begin{document}
\maketitle

This is a quick introduction to the models presented in ``A null model
for quantifying the geometric effect of habitat subdivision on species
diversity''. The R code used in the paper are in file `functions.R' and
the aim here is to illustrate their use. The R-code provides a brief
explanation of the arguments so I don't go into that too much.

Basically there are two parts to the paper, a validation against
empirical data and a theoretical exploration of the implications. I'll
give an example in the reverse order here.

\hypertarget{estimating-diversity}{%
\subsection{Estimating diversity}\label{estimating-diversity}}

First we need some data - if we assume a value for the community scaling
parameter, we only need the species abundance distribution (SAD). Here
I've used data from BCI - that is the
\href{https://repository.si.edu/handle/10088/20925}{Barro Colorado
Island} forest dynamics plot. I use the 2005 census, with 211845 living
stems and 301 species.

Of course, the SAD could be simulated assuming any number of different
abundance distributions\ldots{}

\begin{Shaded}
\begin{Highlighting}[]
\FunctionTok{load}\NormalTok{(}\StringTok{\textquotesingle{}BCI\_SAD.RData\textquotesingle{}}\NormalTok{)}
\end{Highlighting}
\end{Shaded}

The models are all in `functions.R'.

\begin{Shaded}
\begin{Highlighting}[]
\FunctionTok{source}\NormalTok{(}\StringTok{"functions.R"}\NormalTok{)}
\FunctionTok{args}\NormalTok{(predSS.NB)}
\end{Highlighting}
\end{Shaded}

\begin{verbatim}
## function (area, sad, cpar, m, tota = 5e+05) 
## NULL
\end{verbatim}

The arguments are common to the functions and represent:\\
--area = sampling grain - in the same units as tota.\\
--sad = species abundance distribution over the study extent\\
--cpar = community level scaling parameter at sampling grain `area'\\
--tota = extent over which the number of individuals in the species
abundance distribution were counted (defaults to 50 ha)

To predict the diversity of a set of samples from we first calculate
zeta diversity and then plug it into the formula derived in Hui \&
McGeoch (2014).

\begin{Shaded}
\begin{Highlighting}[]
\NormalTok{z20 }\OtherTok{\textless{}{-}} \FunctionTok{predSS.NB}\NormalTok{(}\AttributeTok{area=}\DecValTok{200}\NormalTok{, }\AttributeTok{sad=}\NormalTok{bciSAD, }\AttributeTok{cpar =} \FloatTok{0.88}\NormalTok{, }\AttributeTok{m =} \DecValTok{20}\NormalTok{)}
\NormalTok{z20}
\end{Highlighting}
\end{Shaded}

\begin{verbatim}
##  [1] 32.328150 14.259605  9.413038  7.359025  6.256913  5.568578  5.091642
##  [8]  4.736522  4.458420  4.232628  4.044371  3.884204  3.745752  3.624516
## [15]  3.517213  3.421376  3.335106  3.256920  3.185635  3.120300
\end{verbatim}

We can calculate the total number of species (which is a kind of gamma
diversity)

\begin{Shaded}
\begin{Highlighting}[]
\FunctionTok{gam.fn}\NormalTok{(z20)}
\end{Highlighting}
\end{Shaded}

\begin{verbatim}
## [1] 141.8386
\end{verbatim}

Or the number of species in only a single one of those 20 patches

\begin{Shaded}
\begin{Highlighting}[]
\FunctionTok{spe.fn}\NormalTok{(z20)}
\end{Highlighting}
\end{Shaded}

\begin{verbatim}
## [1] 45.9334
\end{verbatim}

\hypertarget{exploring-the-sloss-question}{%
\subsection{Exploring the SLOSS
question}\label{exploring-the-sloss-question}}

So, if we are interested in the relative number of species in one large
patch for increasing subdivision (noting all the model caveats about
independence and distance decay described in the paper) all we need is a
scaling relationship for c.~We'll use the empirical estimate from the
data in the paper. Where the c parameter for area a was equal to:

c\_a = c\_400*(a/400)\^{}0.28.

So we can compare the species in, say 20 20 x 20 m quadrats with those
in 1 x 8000 m2 (i.e., same total area) as follows

\begin{Shaded}
\begin{Highlighting}[]
\NormalTok{c.sl }\OtherTok{=} \FloatTok{1.06}\SpecialCharTok{*}\NormalTok{(}\DecValTok{8000}\SpecialCharTok{/}\DecValTok{400}\NormalTok{)}\SpecialCharTok{\^{}}\FloatTok{0.28}  \CommentTok{\# single large = 1 x 8000 m2}
\NormalTok{c.ss }\OtherTok{=} \FloatTok{1.06}\SpecialCharTok{*}\NormalTok{(}\DecValTok{800}\SpecialCharTok{/}\DecValTok{400}\NormalTok{)}\SpecialCharTok{\^{}}\FloatTok{0.28}  \CommentTok{\# several small = 10 x 800 m2}

\NormalTok{sr.sl }\OtherTok{\textless{}{-}} \FunctionTok{predSS.NB}\NormalTok{(}\AttributeTok{area=}\DecValTok{8000}\NormalTok{, }\AttributeTok{sad=}\NormalTok{bciSAD, }\AttributeTok{cpar =}\NormalTok{ c.sl, }\AttributeTok{m=}\DecValTok{1}\NormalTok{)}
\NormalTok{sr.ss }\OtherTok{\textless{}{-}} \FunctionTok{gam.fn}\NormalTok{(}\FunctionTok{predSS.NB}\NormalTok{(}\AttributeTok{area=}\DecValTok{800}\NormalTok{, }\AttributeTok{sad=}\NormalTok{bciSAD, }\AttributeTok{cpar =}\NormalTok{ c.ss, }\AttributeTok{m=}\DecValTok{10}\NormalTok{))}
\NormalTok{sr.ss; sr.sl}
\end{Highlighting}
\end{Shaded}

\begin{verbatim}
## [1] 168.3132
\end{verbatim}

\begin{verbatim}
## [1] 157.9507
\end{verbatim}

So \textasciitilde6\% increase in richness is expected if the single
patch was subdivided into 10 independent patches.

\hypertarget{comparing-observed-and-predicted}{%
\section{Comparing observed and
predicted}\label{comparing-observed-and-predicted}}

\hypertarget{negative-binomial}{%
\subsection{1. negative binomial}\label{negative-binomial}}

For comparison with empirical data I'll again use BCI data. The object
`calstats' contains values calculated from 100 repeat samples of 20
randomly positioned 20 x 20 m quadrats.

\begin{Shaded}
\begin{Highlighting}[]
\FunctionTok{load}\NormalTok{(}\StringTok{"calstats.RData"}\NormalTok{)}
\FunctionTok{names}\NormalTok{(calstats)}
\end{Highlighting}
\end{Shaded}

\begin{verbatim}
## [1] "zeta"  "spe"   "beta"  "gamma"
\end{verbatim}

We have four objects, each giving the mean value from the 100 samples
from BCI along with 95\% sampling intervals:

--`zeta' is zeta diversity (average number of species shared) in 1, 2,
\ldots{} , 20 samples.\\
--`spe' is the mean number of species found in only 1 quadrat (single
patch endemics)\\
--`beta' is the mean Sorensen dissimilarity\\
--`gamma' is the total species richness

We can predict zeta using the non-random shared species equation and
then use this to predict the diversity patterns from the zeta components
(this is all derived in Hui \& McGeoch 2014).

First, we need to estimate the community scaling parameter for the 20 x
20 sampling grain.

\begin{Shaded}
\begin{Highlighting}[]
\NormalTok{alpha.div }\OtherTok{=}\NormalTok{ calstats}\SpecialCharTok{$}\NormalTok{zeta[}\DecValTok{1}\NormalTok{,}\DecValTok{1}\NormalTok{];alpha.div }\CommentTok{\# mean number of spp shared in 1 sample = alpha diversity}
\end{Highlighting}
\end{Shaded}

\begin{verbatim}
## [1] 49.0575
\end{verbatim}

Once we have this, we can use function fit.c.NB() to estimate the
parameter. (NB: If you want to get a scaling relationship for c, repeat
the sampling and fitting steps at a few sampling grains, then fit Eq 8
in the main text).

\begin{Shaded}
\begin{Highlighting}[]
\NormalTok{cpar.fit }\OtherTok{\textless{}{-}} \FunctionTok{fitc.NB}\NormalTok{(}\AttributeTok{obs =}\NormalTok{ alpha.div, }\AttributeTok{area =} \DecValTok{400}\NormalTok{, }\AttributeTok{sad =}\NormalTok{ bciSAD, }\AttributeTok{tota=} \DecValTok{500000}\NormalTok{, }\AttributeTok{low=}\DecValTok{0}\NormalTok{, }\AttributeTok{upp=}\DecValTok{100}\NormalTok{)}
\NormalTok{cpar.fit}\SpecialCharTok{$}\NormalTok{cpar}
\end{Highlighting}
\end{Shaded}

\begin{verbatim}
## [1] 1.038374
\end{verbatim}

This is the value for the community scaling parameter, c, at this
sampling grain. Now we just plug it into fitSS.NB() to estimate zeta
diversity in the 20 samples\ldots{}

\begin{Shaded}
\begin{Highlighting}[]
\NormalTok{zeta.est }\OtherTok{\textless{}{-}} \FunctionTok{predSS.NB}\NormalTok{(}\AttributeTok{area =} \DecValTok{400}\NormalTok{, }\AttributeTok{sad =}\NormalTok{ bciSAD, }\AttributeTok{cpar =}\NormalTok{ cpar.fit}\SpecialCharTok{$}\NormalTok{cpar, }\AttributeTok{m =} \DecValTok{20}\NormalTok{, }\AttributeTok{tota =} \DecValTok{500000}\NormalTok{)}
\NormalTok{zeta.est}
\end{Highlighting}
\end{Shaded}

\begin{verbatim}
##  [1] 49.057500 25.730625 18.149225 14.482429 12.358583 10.989532 10.039852
##  [8]  9.344055  8.811930  8.390611  8.047357  7.760973  7.517223  7.306243
## [15]  7.121013  6.956415  6.808641  6.674802  6.552672  6.440503
\end{verbatim}

\ldots{} and compare with the empirical data.

\begin{Shaded}
\begin{Highlighting}[]
\FunctionTok{plot}\NormalTok{(}\DecValTok{1}\SpecialCharTok{:}\DecValTok{20}\NormalTok{, calstats}\SpecialCharTok{$}\NormalTok{zeta[,}\DecValTok{1}\NormalTok{], }\AttributeTok{type=}\StringTok{"l"}\NormalTok{,}\AttributeTok{ylim=}\FunctionTok{c}\NormalTok{(}\DecValTok{0}\NormalTok{,}\DecValTok{60}\NormalTok{),}\AttributeTok{ylab=} \StringTok{\textquotesingle{}Zeta diversity\textquotesingle{}}\NormalTok{, }\AttributeTok{xlab=}\StringTok{"Order of zeta"}\NormalTok{)}
\FunctionTok{lines}\NormalTok{(}\DecValTok{1}\SpecialCharTok{:}\DecValTok{20}\NormalTok{, calstats}\SpecialCharTok{$}\NormalTok{zeta[,}\DecValTok{2}\NormalTok{], }\AttributeTok{lty=} \DecValTok{2}\NormalTok{)}
\FunctionTok{lines}\NormalTok{(}\DecValTok{1}\SpecialCharTok{:}\DecValTok{20}\NormalTok{, calstats}\SpecialCharTok{$}\NormalTok{zeta[,}\DecValTok{3}\NormalTok{], }\AttributeTok{lty=} \DecValTok{2}\NormalTok{)}
\FunctionTok{points}\NormalTok{(}\DecValTok{1}\SpecialCharTok{:}\DecValTok{20}\NormalTok{, zeta.est)}
\FunctionTok{legend}\NormalTok{(}\DecValTok{15}\NormalTok{,}\DecValTok{50}\NormalTok{,}\AttributeTok{legend=}\FunctionTok{c}\NormalTok{(}\StringTok{"Predicted"}\NormalTok{,}\StringTok{"Observed"}\NormalTok{),}\AttributeTok{pch=}\FunctionTok{c}\NormalTok{(}\DecValTok{1}\NormalTok{,}\ConstantTok{NA}\NormalTok{),}\AttributeTok{lty=}\FunctionTok{c}\NormalTok{(}\ConstantTok{NA}\NormalTok{,}\DecValTok{1}\NormalTok{))}
\end{Highlighting}
\end{Shaded}

\includegraphics{Worked-examples_files/figure-latex/unnamed-chunk-11-1.pdf}

Not perfect, but it's an approximate model. Let's see how it does for
the diversity metrics.

First total species richness of the samples:

\begin{Shaded}
\begin{Highlighting}[]
\FunctionTok{gam.fn}\NormalTok{(zeta.est)}
\end{Highlighting}
\end{Shaded}

\begin{verbatim}
## [1] 171.0906
\end{verbatim}

\begin{Shaded}
\begin{Highlighting}[]
\FunctionTok{unlist}\NormalTok{(calstats}\SpecialCharTok{$}\NormalTok{gamma)}
\end{Highlighting}
\end{Shaded}

\begin{verbatim}
##   mean   hi95  low95 
## 175.62 188.00 163.00
\end{verbatim}

\begin{Shaded}
\begin{Highlighting}[]
\NormalTok{(}\FloatTok{171.0905{-}175.62}\NormalTok{)}\SpecialCharTok{/}\FloatTok{175.62}
\end{Highlighting}
\end{Shaded}

\begin{verbatim}
## [1] -0.02579148
\end{verbatim}

Number of species found in a single patch:

\begin{Shaded}
\begin{Highlighting}[]
\FunctionTok{spe.fn}\NormalTok{(zeta.est)}
\end{Highlighting}
\end{Shaded}

\begin{verbatim}
## [1] 46.0355
\end{verbatim}

\begin{Shaded}
\begin{Highlighting}[]
\FunctionTok{unlist}\NormalTok{(calstats}\SpecialCharTok{$}\NormalTok{spe)}
\end{Highlighting}
\end{Shaded}

\begin{verbatim}
##  mean  hi95 low95 
## 43.96 56.00 35.00
\end{verbatim}

\begin{Shaded}
\begin{Highlighting}[]
\NormalTok{(}\FloatTok{46.0355{-}43.96}\NormalTok{)}\SpecialCharTok{/}\FloatTok{43.96}
\end{Highlighting}
\end{Shaded}

\begin{verbatim}
## [1] 0.04721338
\end{verbatim}

And Sorensen dissimilarity:

\begin{Shaded}
\begin{Highlighting}[]
\FunctionTok{bd.fn}\NormalTok{(zeta.est)}
\end{Highlighting}
\end{Shaded}

\begin{verbatim}
## [1] 0.4755007
\end{verbatim}

\begin{Shaded}
\begin{Highlighting}[]
\FunctionTok{unlist}\NormalTok{(calstats}\SpecialCharTok{$}\NormalTok{beta)}
\end{Highlighting}
\end{Shaded}

\begin{verbatim}
##      mean      hi95     low95 
## 0.4936184 0.6155491 0.3616667
\end{verbatim}

\begin{Shaded}
\begin{Highlighting}[]
\NormalTok{(}\FloatTok{0.4755007{-}0.4936184}\NormalTok{)}\SpecialCharTok{/}\FloatTok{0.4936184} 
\end{Highlighting}
\end{Shaded}

\begin{verbatim}
## [1] -0.03670386
\end{verbatim}

One of the things about the null models when compared with real data is
that the diversity measures calculated from zeta components tend to be
much closer to the empirical values, here all less than 5\% from the
empirical estimate. This seems to remain true, even when the zeta
components themselves fell outside the 95\% empirical sampling limits
(the regularly distributed community in the SI is a good example).

\hypertarget{finite-negative-binomial}{%
\subsection{2. Finite negative
binomial}\label{finite-negative-binomial}}

For completeness, here's the fitting for the FNB model. It takes a long
time and does not seem to provide an improvement that warrants this in
my view.

I'm pretty agricultural in my R skills, so I'm sure others will be able
to work out a means to speed this up\ldots{}

\begin{Shaded}
\begin{Highlighting}[]
\CommentTok{\# fit c as with the neg bin}
\NormalTok{fnb.c }\OtherTok{\textless{}{-}} \FunctionTok{fitc.FNB}\NormalTok{(}\AttributeTok{obs =}\NormalTok{ alpha.div, }\AttributeTok{area =}\DecValTok{400}\NormalTok{, }\AttributeTok{sad =}\NormalTok{ bciSAD, }\AttributeTok{tota=} \DecValTok{500000}\NormalTok{, }\AttributeTok{low=}\DecValTok{0}\NormalTok{, }\AttributeTok{upp=}\DecValTok{10}\NormalTok{)}\SpecialCharTok{$}\NormalTok{cpar}
\end{Highlighting}
\end{Shaded}

\begin{verbatim}
## Loading required package: Rmpfr
\end{verbatim}

\begin{verbatim}
## Loading required package: gmp
\end{verbatim}

\begin{verbatim}
## 
## Attaching package: 'gmp'
\end{verbatim}

\begin{verbatim}
## The following objects are masked from 'package:base':
## 
##     %*%, apply, crossprod, matrix, tcrossprod
\end{verbatim}

\begin{verbatim}
## C code of R package 'Rmpfr': GMP using 64 bits per limb
\end{verbatim}

\begin{verbatim}
## 
## Attaching package: 'Rmpfr'
\end{verbatim}

\begin{verbatim}
## The following object is masked from 'package:gmp':
## 
##     outer
\end{verbatim}

\begin{verbatim}
## The following objects are masked from 'package:stats':
## 
##     dbinom, dgamma, dnbinom, dnorm, dpois, dt, pnorm
\end{verbatim}

\begin{verbatim}
## The following objects are masked from 'package:base':
## 
##     cbind, pmax, pmin, rbind
\end{verbatim}

\begin{Shaded}
\begin{Highlighting}[]
\NormalTok{fnb.c}
\end{Highlighting}
\end{Shaded}

\begin{verbatim}
## [1] 1.021112
\end{verbatim}

\begin{Shaded}
\begin{Highlighting}[]
\CommentTok{\# and predict}
\NormalTok{zeta.est.fnb }\OtherTok{\textless{}{-}} \FunctionTok{predSS.FNB}\NormalTok{(}\AttributeTok{sad =}\NormalTok{ bciSAD, }\AttributeTok{cpar =}\NormalTok{ fnb.c, }\AttributeTok{area =} \DecValTok{400}\NormalTok{, }\AttributeTok{tota =} \DecValTok{500000}\NormalTok{, }\AttributeTok{m =} \DecValTok{20}\NormalTok{)}
\NormalTok{zeta.est.fnb}
\end{Highlighting}
\end{Shaded}

\begin{verbatim}
##  [1] 49.057500 25.814989 18.225898 14.549437 12.417853 11.042869 10.088647
##  [8]  9.389351  8.854502  8.431042  8.086087  7.798335  7.553472  7.341577
## [15]  7.155583  6.990340  6.842013  6.707694  6.585139  6.472591
\end{verbatim}

How do the FNB and NB shared species estimates compare?

\begin{Shaded}
\begin{Highlighting}[]
\FunctionTok{plot}\NormalTok{(}\DecValTok{1}\SpecialCharTok{:}\DecValTok{20}\NormalTok{, calstats}\SpecialCharTok{$}\NormalTok{zeta[,}\DecValTok{1}\NormalTok{], }\AttributeTok{type=}\StringTok{"l"}\NormalTok{,}\AttributeTok{ylim=}\FunctionTok{c}\NormalTok{(}\DecValTok{0}\NormalTok{,}\DecValTok{60}\NormalTok{),}\AttributeTok{ylab=} \StringTok{\textquotesingle{}Zeta diversity\textquotesingle{}}\NormalTok{, }\AttributeTok{xlab=}\StringTok{"Order of zeta"}\NormalTok{)}
\FunctionTok{lines}\NormalTok{(}\DecValTok{1}\SpecialCharTok{:}\DecValTok{20}\NormalTok{, calstats}\SpecialCharTok{$}\NormalTok{zeta[,}\DecValTok{2}\NormalTok{], }\AttributeTok{lty=} \DecValTok{2}\NormalTok{)}
\FunctionTok{lines}\NormalTok{(}\DecValTok{1}\SpecialCharTok{:}\DecValTok{20}\NormalTok{, calstats}\SpecialCharTok{$}\NormalTok{zeta[,}\DecValTok{3}\NormalTok{], }\AttributeTok{lty=} \DecValTok{2}\NormalTok{)}
\FunctionTok{points}\NormalTok{(}\DecValTok{1}\SpecialCharTok{:}\DecValTok{20}\NormalTok{, zeta.est)}
\FunctionTok{points}\NormalTok{(}\DecValTok{1}\SpecialCharTok{:}\DecValTok{20}\NormalTok{, zeta.est.fnb, }\AttributeTok{col=}\DecValTok{2}\NormalTok{, }\AttributeTok{pch=}\DecValTok{2}\NormalTok{)}
\FunctionTok{legend}\NormalTok{(}\DecValTok{12}\NormalTok{,}\DecValTok{50}\NormalTok{,}\AttributeTok{legend=}\FunctionTok{c}\NormalTok{(}\StringTok{"Predicted NB"}\NormalTok{,}\StringTok{"Predicted FNB"}\NormalTok{, }\StringTok{"Observed"}\NormalTok{),}\AttributeTok{pch=}\FunctionTok{c}\NormalTok{(}\DecValTok{1}\NormalTok{,}\DecValTok{2}\NormalTok{, }\ConstantTok{NA}\NormalTok{),}\AttributeTok{lty=}\FunctionTok{c}\NormalTok{(}\ConstantTok{NA}\NormalTok{,}\ConstantTok{NA}\NormalTok{, }\DecValTok{1}\NormalTok{), }\AttributeTok{col=}\FunctionTok{c}\NormalTok{(}\DecValTok{1}\NormalTok{,}\DecValTok{2}\NormalTok{,}\DecValTok{1}\NormalTok{))}
\end{Highlighting}
\end{Shaded}

\includegraphics{Worked-examples_files/figure-latex/unnamed-chunk-16-1.pdf}

\begin{Shaded}
\begin{Highlighting}[]
\FunctionTok{plot}\NormalTok{(zeta.est, zeta.est.fnb);}\FunctionTok{abline}\NormalTok{(}\DecValTok{0}\NormalTok{,}\DecValTok{1}\NormalTok{)}
\end{Highlighting}
\end{Shaded}

\includegraphics{Worked-examples_files/figure-latex/unnamed-chunk-17-1.pdf}

\hypertarget{calculate-the-scaling-relationship}{%
\subsubsection{Calculate the scaling
relationship}\label{calculate-the-scaling-relationship}}

To fit a scaling relationship to a stem-mapped plot (or similar), just
need the SAD for the plot and a bunch of species richness estimates at
different grain sizes. There are data from BCI with this information in
the object `validationSamples\_FNB.RData'.

\begin{Shaded}
\begin{Highlighting}[]
\FunctionTok{source}\NormalTok{(}\StringTok{"functions.R"}\NormalTok{)}
\FunctionTok{load}\NormalTok{(}\StringTok{"BCI\_SAD.RData"}\NormalTok{)}
\FunctionTok{load}\NormalTok{(}\StringTok{"validationSamples\_FNB.RData"}\NormalTok{)}
\FunctionTok{names}\NormalTok{(val.sampsFNB[[}\DecValTok{1}\NormalTok{]])}
\end{Highlighting}
\end{Shaded}

\begin{verbatim}
##  [1] "qarea"  "srmod"  "srobs"  "gammod" "gamobs" "spemod" "speobs" "endmod"
##  [9] "endobs" "cest"   "cfit"
\end{verbatim}

We'll need quadrat areas (qarea) observed mean richness for that grain
(srobs). Plus, we can compare the fitted c parameter with the original
estimate (cest) using the finite N (it will be close, but it won't be
the same). The same code could be run with the FNB model but it will
take a while to run\ldots{}

\begin{Shaded}
\begin{Highlighting}[]
\NormalTok{x}\OtherTok{\textless{}{-}}\NormalTok{ val.sampsFNB[[}\DecValTok{1}\NormalTok{]] }\CommentTok{\# bci samples}
\NormalTok{cp }\OtherTok{\textless{}{-}} \FunctionTok{numeric}\NormalTok{()}
\ControlFlowTok{for}\NormalTok{(i }\ControlFlowTok{in} \DecValTok{1}\SpecialCharTok{:}\FunctionTok{nrow}\NormalTok{(x))\{}
\NormalTok{  area }\OtherTok{=}\NormalTok{ x}\SpecialCharTok{$}\NormalTok{qarea[i] }\CommentTok{\# quadrat area (grain size)}
\NormalTok{  obs }\OtherTok{=}\NormalTok{ x}\SpecialCharTok{$}\NormalTok{srobs[i] }\CommentTok{\# mean richness at this grain}
\NormalTok{  cp[i] }\OtherTok{\textless{}{-}} \FunctionTok{fitc.NB}\NormalTok{(}\AttributeTok{obs =}\NormalTok{ obs, }\AttributeTok{area =}\NormalTok{ area, }\AttributeTok{sad =}\NormalTok{ bciSAD, }\AttributeTok{tota =} \FloatTok{5e+05}\NormalTok{, }\AttributeTok{low =} \DecValTok{0}\NormalTok{, }\AttributeTok{upp =} \DecValTok{100}\NormalTok{)}\SpecialCharTok{$}\NormalTok{cpar}
  \FunctionTok{print}\NormalTok{(}\FunctionTok{c}\NormalTok{(cp[i], x}\SpecialCharTok{$}\NormalTok{cest[i])) }\CommentTok{\# compare with original model}
\NormalTok{\}}
\end{Highlighting}
\end{Shaded}

\begin{verbatim}
## [1] 0.2908141 0.2859559
## [1] 0.5825164 0.5740980
## [1] 0.7998832 0.7875621
## [1] 1.018414 1.001288
## [1] 1.205550 1.182773
## [1] 1.378862 1.349404
## [1] 1.510469 1.473620
## [1] 1.652632 1.607238
## [1] 1.769279 1.714719
## [1] 1.864122 1.799919
\end{verbatim}

Now we have the list of areas and the value of the scaling parameter. We
just need to fit a power function.

\begin{Shaded}
\begin{Highlighting}[]
\NormalTok{avec }\OtherTok{\textless{}{-}}\NormalTok{ x}\SpecialCharTok{$}\NormalTok{qarea}\SpecialCharTok{/}\NormalTok{x}\SpecialCharTok{$}\NormalTok{qarea[}\DecValTok{4}\NormalTok{] }\CommentTok{\# use 400 m as base level and convert grain to relative grain}
\NormalTok{k0 }\OtherTok{\textless{}{-}}\NormalTok{ cp[}\DecValTok{4}\NormalTok{] }\CommentTok{\# fitted scaling parameter}
\NormalTok{y }\OtherTok{\textless{}{-}}\NormalTok{ cp }
\NormalTok{nly }\OtherTok{\textless{}{-}} \FunctionTok{nls}\NormalTok{(y }\SpecialCharTok{\textasciitilde{}}\NormalTok{ k0}\SpecialCharTok{*}\NormalTok{avec}\SpecialCharTok{\^{}}\NormalTok{z, }\AttributeTok{start=}\FunctionTok{list}\NormalTok{(}\AttributeTok{z=}\FloatTok{0.2}\NormalTok{))}
\NormalTok{pry }\OtherTok{\textless{}{-}} \FunctionTok{predict}\NormalTok{(nly)}
\FunctionTok{with}\NormalTok{(x, }\FunctionTok{plot}\NormalTok{(qarea, cp))}
\FunctionTok{lines}\NormalTok{(x}\SpecialCharTok{$}\NormalTok{qarea, pry)}
\end{Highlighting}
\end{Shaded}

\includegraphics{Worked-examples_files/figure-latex/unnamed-chunk-20-1.pdf}
Once we have this relationship, we can explore subdivision of area by
adjusting c to the correct grain (as was done in Fig. 5 - see
scr\_simsSubDiv.R for that code).

\hypertarget{reference}{%
\subsection{Reference}\label{reference}}

Hui, C. and McGeoch, M. (2014) Zeta diversity as a concept and metric
that unifies incidence-based biodiversity patterns American Naturalist
2014 Vol. 184 Issue 5 Pages 684-694

\end{document}
